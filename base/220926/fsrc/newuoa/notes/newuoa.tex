%\documentclass[11pt,a4paper]{article}  % Use this line if this document will be released
\documentclass[11pt,a4paper,draft]{article}  % Use this line if this document is a draft
\usepackage{ifdraft}


%% Bibliography
\newcommand{\bibfile}{\jobname.bib}  % Name of the BibTeX file.
\newcommand{\universalbib}{/home/zaikunzhang/Bureau/bibliographie/ref.bib}
\ifdraft{\IfFileExists{\universalbib}{\renewcommand{\bibfile}{\universalbib}}{}}{}
\newcommand{\iscite}{0}  % \iscite=0: without citation; \iscite=1: with citation
\def\oldcite{} \let\oldcite=\cite \def\cite{\renewcommand\iscite{1}\oldcite}


%% Geometry
%\voffset=-1.5cm \hoffset=-1.4cm \textwidth=16cm \textheight=22.0cm  % Luis' setting
\usepackage[a4paper, textwidth=16.0cm, textheight=22.0cm]{geometry}
\renewcommand{\baselinestretch}{1.2}


%% Basic packages
\usepackage{amsmath,amsthm,amssymb,amsfonts,mathtools}
\usepackage{empheq}
\usepackage{color}
\usepackage[bbgreekl]{mathbbol}
\DeclareSymbolFontAlphabet{\mathbbm}{bbold}
\DeclareSymbolFontAlphabet{\mathbb}{AMSb}
\usepackage{bbm}
\usepackage{upgreek}
\usepackage{accents}
\usepackage{xspace}
\usepackage{rotating}
\usepackage{multirow,booktabs}
\usepackage[en-US]{datetime2}


%% Graph, tikz and pgf
%\usepackage{subfigure}
\setlength{\unitlength}{1mm}
% The \unitlength command is a Length command. It defines the units used in the Picture Environment.
\usepackage{graphicx}
%\usepackage{tikz,tikzscale,pgf,pgfarrows,pgfnodes,filecontents,tikz-cd}
\usepackage{tikz,tikzscale,pgf}
\usetikzlibrary{arrows,arrows.meta,patterns,positioning,decorations.markings,shapes}
\usepackage{pgfplots}
\usepackage{pgfplotstable}
\usepackage[justification=centering]{caption}
\usepgfplotslibrary{fillbetween}
\pgfplotsset{compat=1.11}

%% Format of the table of content
\usepackage[normalem]{ulem}
\usepackage[toc,page]{appendix}
\renewcommand{\appendixpagename}{\Large{Appendix}}
\renewcommand{\appendixname}{Appendix}
\renewcommand{\appendixtocname}{Appendix}
%\usepackage{sectsty}
\setcounter{tocdepth}{2}


%% Turn off some unharmful warnings
\usepackage{silence}
\WarningFilter{hyperref}{Draft mode on}
\WarningFilter{refcheck}{Unused label}
\WarningFilter{microtype}{`draft' option active}
\WarningFilter{latex}{Writing or overwriting file} % Mute the warning about 'writing/overwriting file'
\WarningFilter{latex}{Writing file} % Mute the warning about 'writing/overwriting file'
\WarningFilter{latex}{Tab has} % Mute the warning about 'Tab has been converted to Blank Space'


%% Enumerate and itemize
\usepackage{eqlist}
\usepackage{enumitem}
\setlist[itemize]{leftmargin=*}
\setlist[enumerate]{leftmargin=*}


%% Hyperref
\definecolor{darkblue}{rgb}{0,0.1,0.5}
\definecolor{darkgreen}{rgb}{0,0.5,0.1}
\definecolor{darkyellow}{rgb}{0.65,0.65,0.01}
\usepackage{hyperref}
\hypersetup{colorlinks, linkcolor=darkblue, anchorcolor=darkblue, citecolor=darkblue, urlcolor=darkblue}
\ifdraft{\usepackage{refcheck}}{} % Check unused labels
\usepackage{url}


%% Algorithm environment
\usepackage[section]{algorithm}
\usepackage{algpseudocode,algorithmicx}
\newcommand{\INPUT}{\textbf{Input}}
\newcommand{\FOR}{\textbf{For}~}
\algrenewcommand\algorithmicrequire{\textbf{Input:}}
\algrenewcommand\algorithmicensure{\textbf{Output:}}
\algrenewcommand\alglinenumber[1]{\normalsize #1.}
\newcommand*\Let[2]{\State #1 $=$ #2}


%% Theorem-like environments
\newtheorem{theorem}{Theorem}%[section]
\newtheorem{conjecture}{Conjecture}%[section]
\newtheorem{corollary}{Corollary}%[section]
\newtheorem{exercise}{Exercise}%[section]
\newtheorem{lemma}{Lemma}%[section]
\newtheorem{problem}{Problem}%[section]
\newtheorem{proposition}{Proposition}%[section]
\newtheorem{remark}{Remark}%[section]
\newtheorem{assumption}{Assumption}%[section]
\newtheorem{example}{Example}%[section]
% Change theoremstyle to ``definition'', which uses textnormal for the text.
\theoremstyle{definition}
\newtheorem{definition}{Definition}%[section]
% proof
\usepackage{xpatch}
\xpatchcmd{\proof}{\itshape}{\normalfont\proofnamefont}{}{}
\newcommand{\proofnamefont}{\bfseries}

%% Equation numbering
\numberwithin{equation}{section}


%% Fine tuning
\usepackage{microtype}
\usepackage[nobottomtitles*]{titlesec} % No section title at the bottom of pages
% Prevent footnote from running to the next page
\interfootnotelinepenalty=10000
% No line break in inline math
\interdisplaylinepenalty=10000
\relpenalty=10000
\binoppenalty=10000
% No widow or orphan lines
\clubpenalty=10000
\widowpenalty=10000
\displaywidowpenalty=10000


% Use @ to put 1 math unit (mu) in math
% See https://nhigham.com/2013/01/07/fine-tuning-spacing-in-latex-equations/
% and also TeXbook p. 155.
\mathcode`@="8000{\catcode`\@=\active\gdef@{\mkern1mu}}


%% Operators, commands
\usepackage{relsize}
\usepackage{nccmath}
\DeclareMathOperator*{\mcap}{\medmath{\bigcap}}
\DeclareMathOperator*{\mcup}{\medmath{\bigcup}}
%\renewcommand{\cap}{\mathsmaller{\bigcap}}
%\renewcommand{\cap}{\mcap}

\newcommand{\ceil}[1]{ {\lceil{#1}\rceil} }
\newcommand{\floor}[1]{ {\lfloor{#1}\rfloor} }

\newcommand{\mbarr}[1]{\mskip.5\thinmuskip\overline{\mskip-.5\thinmuskip {#1} \mskip-.5\thinmuskip}\mskip.5\thinmuskip} % overline short
\newcommand{\mbar}[1]{\,\overline{\!{#1}\!}\,} % overline short italic

\newcommand{\xopt}{\mbar}
\DeclareMathOperator{\xdist}{\updelta}

\DeclareMathOperator{\tr}{tr}
\DeclareMathOperator{\sort}{sort}
\DeclareMathOperator*{\Argmax}{Argmax}
\DeclareMathOperator*{\Argmin}{Argmin}
\DeclareMathOperator*{\argmax}{argmax}
\DeclareMathOperator*{\argmin}{argmin}
\DeclareMathOperator*{\diag}{diag}
\DeclareMathOperator*{\Diag}{Diag}
\DeclareMathOperator{\Span}{span}
\DeclareMathOperator{\med}{med}
\DeclareMathOperator{\essinf}{essinf}
\DeclareMathOperator{\cl}{cl}
\DeclareMathOperator{\vol}{vol}
\DeclareMathOperator{\comp}{C}
\DeclareMathOperator{\sign}{sign}
\DeclareMathOperator{\rank}{rank}
\DeclareMathOperator{\range}{range}
\DeclareMathOperator{\card}{card}
\DeclareMathOperator{\diam}{diam}
\DeclareMathOperator{\dist}{dist}
\newcommand{\disth}{{\operatorname{\updelta_{\sss{H}}}}}
%\newcommand{\ind}[2]{\operatorname{\mathbbm{1}}\;\!\!\!\big(#2\in#1\big)}
\DeclareMathOperator{\ind}{\mathbbm{1}}
\newcommand{\shortd}{\mathsf{S}}
\newcommand{\redrho}{\mathtt{R}}
\newcommand{\impgeo}{\mathsfit{G}}
%\newcommand*{\defeq}{\stackrel{\mbox{\normalfont\tiny{\textnormal{def}}}}{=}}
\newcommand\defeq{\mathrel{\overset{\makebox[0pt]{\mbox{\normalfont\tiny\sffamily def}}}{=}}}

\newcommand{\RR}{\mathbb{R}}
\newcommand{\BB}{\mathcal{B}}
\renewcommand{\SS}{\mathbb{S}}
\newcommand{\ZZ}{\mathbb{Z}}
\newcommand{\NN}{\mathbb{N}}
\newcommand{\FF}{\mathbb{F}}
\newcommand{\CC}{\mathbb{C}}
\newcommand{\Int}{\mathcal{X}}
\newcommand{\Qua}{\mathcal{Q}}
\newcommand{\sss}[1]{{\scriptscriptstyle{#1}}}
\newcommand{\add}{{\textrm{a}}}
\newcommand{\drop}{{\textrm{d}}}
\newcommand{\new}{{\sss{+}}}
\newcommand{\fin}{{\textrm{end}}}
\newcommand{\sK}{{\scriptscriptstyle{K}}}
\newcommand{\sT}{{\scriptscriptstyle{T}}}
\newcommand{\fro}{{\scriptscriptstyle{\textrm{F}}}}
\newcommand{\sob}{{\scriptscriptstyle{\textrm{S}}}}
\newcommand{\pow}{{\scriptscriptstyle{\textrm{P}}}}
\newcommand{\Proj}{\mathrm{Proj}}
\newcommand{\Projs}{\mathrm{P}^\sob}
\newcommand{\trs}{{\scriptscriptstyle{\mathsf{T}}}}
\newcommand{\hmt}{{\scriptscriptstyle{{\mathsf{H}}}}}
\newcommand{\pin}{{\scriptscriptstyle{{\mathsf{+}}}}}
\newcommand{\inv}{{-1}}
\newcommand{\adj}{*}

\newcommand{\cs}{\text{c}}
\newcommand{\hp}{\circ}
\newcommand{\cc}{\sss{\textnormal{C}}}
\newcommand{\dec}{\sss{\textnormal{D}}}
\newcommand{\cauchy}{\sss{\textnormal{C}}}
\newcommand{\scauchy}{\sss{\textnormal{S}}}
\newcommand{\crit}{\textnormal{crit}}
\newcommand{\rsg}{\hat{\partial}}
\newcommand{\gsg}{\partial}
\newcommand{\dom}{\textnormal{dom}}
\newcommand{\tf}{{\textnormal{f}}}
\newcommand{\tg}{{\textnormal{g}}}
\newcommand{\ts}{{\textnormal{s}}}
\newcommand{\st}{\textnormal{s.t.}}
\newcommand{\etc}{{etc.}}
\newcommand{\ie}{{i.e.}}
\newcommand{\eg}{{e.g.}}
\newcommand{\etal}{{et al.}}
\newcommand{\iid}{\text{i.i.d.}}

\newcommand{\me}{\mathrm{e}}
\newcommand{\md}{\mathrm{d}}
\newcommand{\mi}{\mathrm{i}}
\newcommand{\lev}{\mathrm{lev}}
\newcommand{\bA}{\mathbf{A}}
\newcommand{\bx}{\mathbf{u}}
%\newcommand{\bb}{\mathbf{f}}
\newcommand{\bb}{\mathbf{r}}
\newcommand{\nov}{n_{\textnormal{o}}}
\xspaceaddexceptions{]\}}
% tex.stackexchange.com/questions/15252/why-does-xspace-behave-differently-for-parenthesis-vs-braces-brackets
\newcommand{\MATLAB}{\textsc{Matlab}\xspace}
\newcommand{\octave}{\mbox{GNU Octave}\xspace}
\newcommand{\prblm}{\texttt}
\DeclareMathAlphabet{\mathsfit}{T1}{\sfdefault}{\mddefault}{\sldefault}
\SetMathAlphabet{\mathsfit}{bold}{T1}{\sfdefault}{\bfdefault}{\sldefault}
\newcommand{\prbb}{\mathsfit{p}}
\newcommand{\pp}{\mathsf{p}}
\newcommand{\qq}{\mathsf{q}}
\newcommand{\ttt}{\mathsfit{t}}
\newcommand{\tol}{\varepsilon}
\newcommand{\bt}{\mathbf{t}}
\newcommand{\br}{\mathbf{r}}
\newcommand{\dd}{\mathbf{d}}
\newcommand{\ii}{\mathbf{i}}
\newcommand{\jj}{\mathbf{j}}
\newcommand{\xx}{\mathbf{x}}
\renewcommand{\pp}{\mathbf{p}}
\renewcommand{\ggg}{\mathbf{g}}
\newcommand{\GG}{\mathbf{G}}
\renewcommand{\Pr}{\mathbb{P}}

% mathlcal font
\DeclareFontFamily{U}{dutchcal}{\skewchar\font=45 }
\DeclareFontShape{U}{dutchcal}{m}{n}{<-> s*[1.0] dutchcal-r}{}
\DeclareFontShape{U}{dutchcal}{b}{n}{<-> s*[1.0] dutchcal-b}{}
\DeclareMathAlphabet{\mathlcal}{U}{dutchcal}{m}{n}
\SetMathAlphabet{\mathlcal}{bold}{U}{dutchcal}{b}{n}

% mathscr font (supporting lowercase letters)
%\usepackage[scr=dutchcal]{mathalfa}
%\usepackage[scr=esstix]{mathalfa}
%\usepackage[scr=boondox]{mathalfa}
%\usepackage[scr=boondoxo]{mathalfa}
\usepackage[scr=boondoxupr]{mathalfa}
%\newcommand{\model}{\mathscr{h}}
\newcommand{\model}{\tilde{f}}
\newcommand{\rmod}{F}

\newcommand{\Set}[1]{\mathcal{#1}}
\DeclareMathAlphabet{\mathpzc}{OT1}{pzc}{m}{it} % The mathpzc font
\newcommand{\slv}{\mathpzc}
% mathpzc looks great, but it stops working on 19 Feb 2020 for no reason.
%\newcommand{\slv}{\mathscr}
\newcommand{\software}{\texttt}
\DeclareMathOperator{\eff}{\mathsf{e}\;\!}
\DeclareMathOperator{\Eff}{\mathsf{E}\;\!}
\newcommand{\out}{{\text{out}}}


%% Commands for revision
\newcommand{\red}[1]{\textcolor{red}{#1}}
\newcommand{\blue}[1]{\textcolor{blue}{#1}}
\newcommand{\green}[1]{\textcolor{darkgreen}{#1}}
\newcommand{\TYPO}[1]{{\color{orange}{#1}}}
\newcommand{\MISTAKE}[1]{{\color{violet}{#1}}}
\newcommand{\REPHRASE}[1]{{\color{darkgreen}{#1}}}
\newcommand{\REVISION}[1]{{\color{blue}{#1}}}
\newcommand{\REVISIONred}[1]{{\color{red}{#1}}}
\newcommand{\COMMENT}[1]{\textcolor{brown}{{\small{(comment: #1)}}}}
%\newcommand{\REVISION}[1]{#1}
%\newcommand{\REVISIONred}[1]{#1}


%%%%%%%%%%%%%%%%%%%%%%%%%%%%%%%%%%%%%%%%%%%%%%%%%%%%%%%%%%%%%%%%%%%%%%%%%%%%%%%%%%%%%%%%%%%%%%%%%%%%
\title{Notes on NEWUOA}

\date{\DTMnow}

\author{Zaikun Zhang
    \thanks{Hong Kong Polytechnic University, \url{zaikun.zhang@polyu.edu.hk}}
    %\and
    %Author2
    %\thanks{Information2}
}


\begin{document}

\maketitle

%\begin{abstract}
%\end{abstract}

%\textbf{Keywords}: Keyword1, Keyword2
%%%%%%%%%%%%%%%%%%%%%%%%%%%%%%%%%%%%%%%%%%%%%%%%%%%%%%%%%%%%%%%%%%%%%%%%%%%%%%%%%%%%%%%%%%%%%%%%%%%%

We use~$\Int_k$ to denote the set of interpolation points at iteration~$k$.
In addition, we define~$\xopt{\Int_k}$ to be a point in~$\Int_k$ such that
\begin{equation}
    \label{eq:xopt}
    f(\xopt{\Int_k}) = \min\{f(x) \mathrel{:} x\in \Int_k\}.
\end{equation}
If multiple points attain the minimum, then we take the earliest one visited by the algorithm.
Meanwhile, we define
\begin{equation}
    \label{eq:xdist}
    \xdist(\Int_k) = \max\{\|x-\xopt{\Int_k}\| \mathrel{:} x\in \Int_k\}.
\end{equation}

Let~$\Qua$ be the linear space of all the $n$-variable polynomials with degree at most two. The
elements in~$\Qua$ will be referred to as quadratics, even though they may be linear or constant.
Given~$\Int_k$, we denote
\begin{equation}
    \label{eq:quak}
    \Qua_k = \{Q\in\Qua \mathrel{:} Q(x) = f(x)~\text{for}~x\in\Int_k\},
\end{equation}
which is the set of quadratics that interpolate~$f$ at~$\Int_k$. Note that~$\Qua_k$ is an affine
subset of~$\Qua$.
%Similarly,~$\Qua_k^\new = \{Q\in\Qua \mathrel{:} Q(x) = f(x)~\text{for}~x\in\Int_k^\new\}$.
\begin{equation}
    \label{eq:snorm}
    \|Q\|_\sob = \|\nabla^2 Q\|_\fro
\end{equation}
\begin{equation*}
    \Proj_\mathcal{C}^\pow \quad
    \Projs_{\mathcal{C}}(\Phi) = \argmin\{\|Q-\Phi\|_\sob \mathrel{:} Q\in\mathcal{C}\}
\end{equation*}




%\usepackage{algorithm, algpseudocode, algorithmicx}
\begin{algorithm}[htbp!]
    \caption{\label{alg:optim}OPTimization based on Interpolation Models (OPTIM)}
    Input $\Delta_0\in (0,+\infty)$, $\tau>0$, $m\in \{1, 2, \dots, (n+1)(n+2)/2\}$,
    and~$\Int_0\subset \RR^n$ with~$|\Int_0|=m$ and~$\kappa(\Int_0) \le \kappa_0$. Set~$k=0$.
    \begin{algorithmic}[1]
        \State \textbf{Model construction}.
        Pick~$Q_k \in \Qua_k$.
        \State \textbf{Trust-region step evaluation}.
        Calculate
        \begin{equation}
         \label{eq:xadd}
         x_k^\add \approx \argmin\{Q_k(x)\mathrel{:} \|x-\xopt{\Int_k}\|\le \Delta_k\}.
        \end{equation}
        If~$\|x_k^\add-\xopt{\Int_k}\| < \alpha\Delta_k$, then set~$\Delta_{k+1} = \theta \Delta_k$;
        otherwise, update~$\Delta_k$ to~$\Delta_{k+1}$ according to~$r_k = [f(\xopt{\Int_k})
        - f(x_k^\add)]/[Q_k(\xopt{\Int_k}) - Q_k(x_k^\add)]$.
        \State \textbf{Interpolation set update}.
        Let
        \begin{equation}
            \label{eq:xdrop}
            x_k^\drop \approx \argmin\{\kappa(\Int_k\cup x_k^\add\setminus x) \mathrel{:} x \in
            \Int_k\}.
        \end{equation}
        If~$\|x_k^\add-\xopt{\Int_k}\| \ge \alpha\Delta_k$
        and either $r_k>\eta_0$ or $\kappa(\Int_k\cup x_k^\add \setminus x_k^\drop) \le \kappa_0$,
        then set~$\Int_{k+1} = \Int_k\cup x_k^\add\setminus x_k^\drop$.
        \State \textbf{Geometry improvement}.
        If~$\|x_k^\add-\xopt{\Int_k}\|< \alpha \Delta_k$, or if~$r_k\le \eta_0$ and $\kappa(\Int_k
        \cup x_k^\add\setminus x_k^\drop) > \kappa_0$, then set~$\Int_{k+1} = \Int_k\cup
        y_k^\add\setminus y_k^\drop$ with
          \begin{align}
              \label{eq:ydrop}
              y_k^\drop &= \argmax\{\|y-\xopt{\Int_k}\| \mathrel{:} y \in \Int_k\}, \\
              \label{eq:yadd}
              y_k^\add &\approx \argmin\{\kappa(\Int_k \cup y \setminus y_k^\drop) \mathrel{:}
                  \|y-\xopt{\Int_k}\|\le
              \Delta_k\}.
          \end{align}
    \end{algorithmic}
\end{algorithm}


%\usepackage{algorithm, algpseudocode, algorithmicx}
\begin{algorithm}[htbp!]
    \caption{\label{alg:newuoa}NEWUOA}
    Input $\Delta_0\in (0,+\infty)$, $\rho_{\fin}>0$, $m\in \{n+2, n+3, \dots, (n+1)(n+2)/2\}$,
    and~$\Int_0\subset \RR^n$ with~$|\Int_0|=m$ and~$\kappa(\Int_0) \le \kappa_0$.
    Define~$Q_0 = \Projs_{\Qua_0}(0)$.
    Set~$k=0$.
    \begin{algorithmic}[1]
        \State \textbf{Trust-region step evaluation}.
        Set
        \begin{align}
             \label{eq:xaddn}
             x_k^\add \approx &\argmin\{Q_k(x)\mathrel{:} \|x-\xopt{\Int_k}\|\le \Delta_k\}.\\
            \label{eq:shortd}
            \shortd = &\ind(\|x_k^\add-\xopt{\Int_k}\| < \rho_k/2),\\
            \label{eq:redrho}
            \redrho = & \ind(\shortd=1\text{ and the errors in recent models are small}).
        \end{align}
        If~$\redrho = 1$, then let~$\Int_{k+1} = \Int_k$, $Q_{k+1} = Q_k$, and go to step 4.
        If~$\redrho = 0$ and~$\shortd = 1$, then set~$\Delta_{k+1} = \max\{\Delta_k/10,\, \rho_k\}$.
        If~$\shortd = 0$, then evaluate $r_k = [f(\xopt{\Int_k})-f(x_k^\add)]/[Q_k(\xopt{\Int_k})-Q_k(x_k^\add)]$ and update~$\Delta_k$ to~$\Delta_{k+1}$ according to~$r_k$.
        \State \textbf{Interpolation set update}.
        Let
        \begin{equation}
            \label{eq:xdropn}
            x_k^\drop \approx \argmin\{\kappa(\Int_k\cup x_k^\add\setminus x) \mathrel{:} x \in
            \Int_k\}.
        \end{equation}
        If~$\shortd = 0$ and either~$r_k>0$ or $\kappa(\Int_k\cup x_k^\add \setminus x_k^\drop) \le
        \kappa_0$, then set
        \begin{equation}
            \label{eq:updateq1}
            \Int^{\new}_{k} = \Int_k\cup x_k^\add\setminus x_k^\drop, \quad
            {Q}^{\new}_k = \Projs_{\Qua_k^\new}(Q_k);
        \end{equation}
        otherwise, $\Int_k^\new = \Int_k$ and~$Q_k^\new =Q_k$.
        \State \textbf{Geometry improvement}.
        Let~$\Delta^{\new}_k = \max\{\min\{\xdist(\Int_k^\new)/10,\, \Delta_{k+1}/2\},\,\rho_{k}\}$,
        and
          \begin{align}
              \label{eq:ydropn}
              y_k^\drop &= \argmax\{\|y-\xopt{\Int^{\new}_k}\| \mathrel{:} y \in \Int^{\new}_k\},\\
              \label{eq:yaddn}
              y_k^\add &\approx \argmin\{\kappa(\Int^{\new}_k \cup y \setminus y_k^\drop) \mathrel{:}
              \|y-\xopt{\Int^{\new}_k}\|\le \Delta^{\new}_k\}.
          \end{align}
        If~$\xdist(\Int^\new_k)\ge 2\Delta_{k+1}$ and either $\shortd = 1$ or $r_k < 1/10$, then set
          \begin{align}
            \label{eq:updateq2}
              \Int_{k+1} = \Int^{\new}_k\cup y_k^\add\setminus y_k^\drop,
              \quad
              Q_{k+1} = \Projs_{\Qua_{k+1}}(Q_k^\new) ;
          \end{align}
          otherwise, $\Int_{k+1} = \Int_k^\new$ and~$Q_{k+1} = Q_k^{\new}$.
          If~$\xdist(\Int_k^{\new}) < 2\Delta_{k+1}$, $\max\{\Delta_{k+1},\, \|x_k^\add
          - \xopt{\Int_k}\|\}\le \rho_k$, and either $\shortd=1$ or~$r_k\le 0$,
          then set~$\redrho$ to~$1$.
       \State \textbf{Resolution enhancement}.
       If~$\redrho = 0$, then set~$\rho_{k+1} = \rho_k$.
       If~$\redrho = 1$ and~$\rho_k >\rho_{\fin}$, then reduce~$\rho_k$ by about a factor of~$10$ to
       obtain~$\rho_{k+1}$ and set~$\Delta_{k+1} = \max\{\rho_{k}/2,\, \rho_{k+1}\}$.
       If~$\redrho = 1$ and~$\rho_k\le\rho_{\fin}$, then exit.
    \end{algorithmic}
\end{algorithm}


%%%%%%%%%%%%%%%%%%%%%%%%%%%%%%%%%%%%%%%%%%%%%%%%%%%%%%%%%%%%%%%%%%%%%%%%%%%%%%%%%%%%%%%%%%%%%%%%%%%%
%% References
% Include references only if \iscite=1, i.e., there are citations.
\ifnum\iscite=1
    \small
    \bibliography{\bibfile}
    \bibliographystyle{plain}
\fi

%% The end
\end{document}
